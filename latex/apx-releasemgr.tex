\chapter{\deprecated{A Release Manager Machine for \zep}}
\label{apx:releasemgr}
At \cern\ we host repositories using OpenSolaris/ZFS, Apache, and Squid servers running on a VMware vSphere virtualized infrastructure.
However, for small VOs similar functionality can be achieved with a single standard SLC5 machine.

So, let's assume you are software manager for the ambitious and imaginary \zep\ experiment driven by the famous and imaginary Aviation Research Institute.
Recently you decided to distribute your software packages with \cvmfs\ and you are in charge of maintaining the releases.
So you acquired an SLC5 machine and an additional harddisk that is supposed to store the repository.
The machine is available from the outside as \texttt{zeppelin-webfs.aviation-research.org}.

\section{Roadmap}
The release manager machine will have three purposes:
\begin{enumerate}
	\item Serve as a template machine where software releases are installed and tested
	\item Store snapshots of the software releases
	\item Publish releases via HTTP
\end{enumerate}
Software releases will be installed in \texttt{/opt/zeppelin}, which is supposed to be a self-contained software directory.
We use the logical volume manger (LVM) in order to create snapshots of different releases.
These snapshots are mounted read-only in \texttt{/srv/www/webfs} and served to the outside by Apache under \url{http://zeppelin-webfs.aviation-research.org/opt/zeppelin}.
This way you are able to install and test new software releases while always serving a consistent repository.

\section{Initial Setup}
We start by installing the necessary software for the rest of the installation procedure.
Make sure, the following SLC5 packages are installed:
\begin{lstlisting}[language=bash]
yum install autoconf automake gcc gcc-c++ pcre-devel curl-devel \
            openssl-devel fuse fuse-devel screen httpd
\end{lstlisting}
Additionally we require \texttt{inotify-tools}.
The tarball can be downloaded from \url{http://inotify-tools.sourceforge.net}.
After untaring, the installation is a straight forward \lstinline{./configure; make; make install}.

We now have the necessary requirements to build and install \cvmfs.
Download the latest tarball from \url{https://cernvm.cern.ch/project/trac/cernvm/downloads} (cf.~Section \ref{sct:start}).
Note that we need to work with the tarball since we require the server part and additional scripts not shipped with the RPM.
After untaring we build and install by
\begin{lstlisting}[language=bash]
./configure --enable-sqlite3-builtin --enable-libcurl-builtin
make
make install
\end{lstlisting}

It's now time to prepare the repository harddisk, which we assume to be located at \texttt{/dev/sdb}.
We start the interactive utility \lstinline{/usr/sbin/parted /dev/sdb} to initialize the disk.
We use the \lstinline{print} command to make sure we see the empty partition table of the correct harddisk.
Also, remember the size of the harddisk as shown by \lstinline{print}.
We create an ms-dos partition table by \lstinline{mklabel} and a primary partition spanning the entire size with \lstinline{mkpart}.
We check again with \lstinline{print} and leave the utility with \lstinline{quit}.
Now there should be \texttt{/dev/sdb} and \texttt{/dev/sdb1}

The next step is to setup LVM in the newly created partition.
At this point, we have to estimate a maximum size of the software repository.
In order to make sure that snapshots never become unusable, each snapshot has to have at least the size of the working repository.
So the size of the harddisk should be large enough to store the working repository and a couple of snapshots.
For this tutorial, let's assume a 200\,GB harddisk and an upper bound of 20\,GB for the repository.
The following commands set up the repository volume:
\begin{lstlisting}[language=bash]
/usr/sbin/pvcreate /dev/sdb1
/usr/sbin/vgcreate vg-zeppelin /dev/sdb1
/usr/sbin/lvcreate -L 20G -n lv-zeppelin vg-zeppelin
/sbin/mkfs.ext3 /dev/vg-zeppelin/lv-zeppelin
\end{lstlisting}
The volume ist supposed to be mounted at \texttt{/srv/cvmfs/zeppelin}.
We create the necessary directory hieves by:
\begin{lstlisting}[language=bash]
mkdir -p /srv/cvmfs/zeppelin # working repository
mkdir -p /srv/www/webfs/zeppelin # Apache document root
mkdir -p /opt/zeppelin # Install and test your software here

# Create repository directory structure
mount  /dev/vg-zeppelin/lv-zeppelin /srv/cvmfs/zeppelin
mkdir -p /srv/cvmfs/zeppelin/pub/catalogs /srv/cvmfs/zeppelin/pub/data \
         /srv/cvmfs/zeppelin/ctrl /srv/cvmfs/zeppelin/shadow
umount /srv/cvmfs/zeppelin
\end{lstlisting}
To mount the volume persistently we edit \texttt{/etc/fstab} and add the lines
\begin{lstlisting}[language=bash]
/dev/vg-zeppelin/lv-zeppelin  /srv/cvmfs/zeppelin  ext3  defaults       0 0
/srv/cvmfs/zeppelin/shadow    /opt/zeppelin        bind  defaults,bind  0 0
\end{lstlisting}
We test the changes with \lstinline{mount -a; mount}.

As last step, we create a user to install and test software.
This user becomes owner of the repository, \ie we run
\begin{lstlisting}[language=bash]
useradd zeppelin
chown -R zeppelin:zeppelin /srv/cvmfs/zeppelin
\end{lstlisting}

\section{Maintaining the Repository}
Your software will be installed in \texttt{/opt/zeppelin}.
This directory has to be under surveillance of \texttt{cvmfs\_journald}.
This utility records all the changes to the \texttt{/opt/zeppelin} directory in order to synchronize it with the \cvmfs\ repository (cf.~Section~\ref{sct:createrepo}).
You can use the \texttt{initd} script in the \texttt{add-ons} directory of the \cvmfs\ tarball.
The important command to startup the utility is
\begin{lstlisting}[language=bash]
screen -dmS cvmfs_journald sudo -u zeppelin sh -c \
 "LD_LIBRARY_PATH=/usr/local/lib:$LD_LIBRARY_PATH /usr/local/bin/cvmfs_journald\
 -w /opt/zeppelin -r /srv/cvmfs/zeppelin/pub -e /srv/cvmfs/zeppelin/inotify_events\
 -l /src/cvmfs/zeppelin/ctrl/log -s /srv/cvmfs/zeppelin/ctrl/in_sync\
 -o /srv/cvmfs/zeppelin/ctrl/ready -a  /srv/cvmfs/zeppelin/ctrl/auth\
 | tee /srv/cvmfs/zeppelin/ctrl/stdout"
\end{lstlisting}
Always make sure that \texttt{cvmfs\_journald} is running before making changes to the repository.
As root you can check with \lstinline{screen -x cvmfs_journald}.

Now switch to user zeppelin by \lstinline{su zeppelin} and install and test your installation.
For now we just create a single file with \lstinline{echo "Hello, CVMFS" > /opt/zeppelin/README}.
For publishing it, we switch back to root, create and LVM snapshot and mount it under \texttt{/srv/www/webfs/zeppelin}.
In the \texttt{add-ons} directory of the \cvmfs\ tarball you'll find the \texttt{cvmfs-publish} shell script that wraps up all the necessary steps.
Listing~\ref{lst:publish} shows the essential commands from that script.
\begin{lstlisting}[caption=Excerpt from \texttt{cvmfs-publish} utility,label=lst:publish,language=bash]
REPOSITORY=$1
COMMAND=$2

switch_snapshot ()
{
  snapshot=$1
  fstabline="/dev/vg-$REPOSITORY/$snapshot /srv/www/webfs/$REPOSITORY \
             ext3 defaults,ro 0 0 #CVMFS_AUTOPUBLISH"
  grep -v CVMFS_AUTOPUBLISH /etc/fstab > /etc/fstab.2
  echo $fstabline >> /etc/fstab.2
  mv /etc/fstab.2 /etc/fstab
  umount /srv/www/webfs/$REPOSITORY 2>/dev/null
  mount -a
}

case $COMMAND in
  list)
    snapshots=`ls /dev/vg-$REPOSITORY/ 2>/dev/null`
    for s in $snapshots
    do
      echo $s
    done    
  ;;
  publish)
    origin="/dev/vg-$REPOSITORY/lv-$REPOSITORY"
    size=`lvs --noheadings -o lv_size --units K $origin`

    touch /opt/$REPOSITORY/.cvmfscatalog 
    sleep 2
    while [ ! -f /srv/cvmfs/$REPOSITORY/ctrl/in_sync ]; do
      sleep 2
    done
    
    timestamp=`date "+%Y%m%d%H%M%S"`
    snapshot=lv-$REPOSITORY-$timestamp
    /usr/sbin/lvcreate -s -L $size -p r -n $snapshot $origin
    
    switch_snapshot $snapshot
  ;;
  switch)
    snapshot=$3
    switch_snapshot $snapshot
  ;;
  remove)
    snapshot=$3
    /usr/sbin/lvremove /dev/vg-$REPOSITORY/$snapshot
  ;;
esac
\end{lstlisting}
So we run \lstinline{cvmfs-publish zeppelin publish} and check with \lstinline{cvmfs-publish zeppelin list}.

\section{Apache Configuration}
We are almost done.
The last step is to make Apache publish \texttt{/srv/www/webfs/zeppelin}.
To do so, we first add the apache user to the zeppelin group by \lstinline{usermod -a -G zeppelin apache}.
Then we drop an Apache configure file in \texttt{/etc/httpd/conf.d/webfs.conf}.
This configuration will create a virtual directory \url{http://zeppelin-webfs.aviation-research.org/opt} without touching the rest of your Apache configuration.
You can find the \texttt{initd} script in the \texttt{add-ons} directory of the \cvmfs\ tarball.
Listing~\ref{lst:apache} shows the essential options from that configuration file.
\begin{lstlisting}[caption=Excerpt from \texttt{webfs.conf} Apache configuration,label= lst:apache,language=bash]
RewriteEngine on

#  - /opt/<experiment> is forced to be lower case 
RewriteMap toLower int:tolower

# Lowering the case for repository names
RewriteRule ^/opt/([A-Za-z0-9\-\.]+)/(.*)$ /opt/${toLower:$1}/pub/catalogs/$2 [PT] 

# Translation URL to real pathname
Alias /opt /srv/www/webfs

# Close access to non-pub directories
<DirectoryMatch /srv/www/webfs/[A-Za-z0-9\-\.]+/(shadow|ctrl)>
    Order allow,deny
    Deny from all
</DirectoryMatch>

<Directory "/srv/www/webfs">
    Options Indexes -MultiViews FollowSymLinks
    AllowOverride All
    Order allow,deny
    Allow from all

    IndexOptions +SuppressDescription +FoldersFirst +VersionSort +SuppressIcon

    EnableMMAP Off
    EnableSendFile Off

    AddType application/x-compress-cvmfs .cvmfschecksum

    FileETag INode MTime Size 
    ExpiresActive On
    ExpiresDefault "access plus 15 minutes"
    ExpiresByType text/html "access plus 5 minutes" 
    ExpiresByType application/x-compress-cvmfs "access" 
</Directory>
\end{lstlisting}

At this point, we have to deal with some of the security measures in SLC5.
In particular, SELinux and iptables will shield you from all the dangers in the world.
Apart from that, they will also shield your users from accessing the repository.
So, as a first step, we tell SELinux that the repository is really supposed to be served by Apache (note that we have to publish again afterwards)
\begin{lstlisting}
chcon -h -R -t httpd_sys_content_t /srv/cvmfs/zeppelin
cvmfs-publish zeppelin publish
\end{lstlisting}
Secondly, in order to let other machines talk to your release manager machine, you have to open TCP port 80 in iptables.
There is a configuration utility to do that without editing the iptables configuration files.
Run \lstinline{system-config-securitylevel} (text based), go to ``customize'' and open WWW for incoming traffic.

\section{Test the Release Manager Machine}
The first step is to open a web browser and look if you can browse through \url{http://zeppelin-webfs.aviation-research.org/opt/zeppelin}.
You should be able to jump to the \texttt{data} directory and in the subdirectories within \texttt{data}.
In case there are any errors, \texttt{/var/log/httpd/access\_log} and \texttt{/var/log/httpd/error\_log} might contain useful information.
For SELinux, looking into \texttt{/var/log/audit/audit.log} might be helpful as well.

To test with the \cvmfs\ client part, we run the following commands:
\begin{lstlisting}
modprobe fuse
mkdir -p /var/cache/cvmfs2/default /mnt/cvmfs
cvmfs2 /mnt/cvmfs http://localhost/opt/zeppelin
cat /mnt/cvmfs/README
umount /mnt/cvmfs
\end{lstlisting}
In case of success, we have just observed a \texttt{Hello, CVMFS}.